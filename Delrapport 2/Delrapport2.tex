\documentclass[a4paper,12pt]{article}

\usepackage[danish]{babel}
\usepackage[utf8]{inputenc}
\usepackage[pdftex]{graphicx}
\usepackage{subfigure}
\usepackage{setspace}
\onehalfspacing
\usepackage{upquote}
\usepackage{color}
\definecolor{listinggray}{gray}{0.9}
\usepackage{listings}
\lstset{
	language=,
	literate=
		{æ}{{\ae}}1
		{ø}{{\o}}1
		{å}{{\aa}}1
		{Æ}{{\AE}}1
		{Ø}{{\O}}1
		{Å}{{\AA}}1,
	backgroundcolor=\color{listinggray},
	tabsize=3,
	rulecolor=,
	basicstyle=\scriptsize,
	aboveskip={1.5\baselineskip},
	columns=fixed,
	showstringspaces=false,
	extendedchars=true,
	breaklines=true,
	prebreak =\raisebox{0ex}[0ex][0ex]{\ensuremath{\hookleftarrow}},
	frame=single,
	showtabs=false,
	showspaces=false,
	showstringspaces=false,
	identifierstyle=\ttfamily,
	keywordstyle=\color[rgb]{0,0,1},
	commentstyle=\color[rgb]{0.133,0.545,0.133},
	stringstyle=\color[rgb]{0.627,0.126,0.941},
}
\usepackage[center,font=small,labelfont=bf,textfont=it]{caption}
\usepackage{enumerate}
\usepackage[numbers]{natbib}
\usepackage{hyperref}
\hypersetup{
	pdfborder = {0 0 0},
    colorlinks,
    citecolor=blue,
    filecolor=blue,
    linkcolor=blue,
    urlcolor=blue
}
\usepackage{parskip}
\setlength{\parindent}{15pt}

\begin{document}

\begin{titlepage}


\newcommand{\HRule}{\rule{\linewidth}{0.5mm}} % Defines a new command for the horizontal lines, change thickness here

\center % Center everything on the page

\textsc{\LARGE Københavns Universitet}\\[1.5cm] % Name of your university/college
\textsc{\Large Projektkursus Systemudvikling 2014}\\[0.5cm] % Major heading such as course name

\HRule \\[0.4cm]
{  \bfseries \large Open Source Bannerreklamesystem \\ \huge EasyAd}\\[0.4cm] % Title of your document
\HRule \\[1.5cm]

\begin{minipage}[t]{0.4\textwidth}
\begin{flushleft} \large
\emph{Projektgruppe:}


Signar Nielsen % Your name
\\
060585
\newline
\\
Aslak Niclasen
\\
060693
\newline
\\
Amanda Bergqvist
\\
171090
\end{flushleft}
\end{minipage}
~
\begin{minipage}[t]{0.4\textwidth}
\begin{flushright} \large
\emph{Instruktor:} \\
Aske Mottelson Clausen % Supervisor's Name
\end{flushright}
\end{minipage}\\[4cm]

{\large 24. april 2014}\\[3cm] % Date, change the \today to a set date if you want to be precise

\end{titlepage}

\tableofcontents %generate table of content


\clearpage %clears float-buffer - inserting those missing - and starts new page


\section{Indledning}
Ifølge en rapport fra Danske Medier Research\footnote{\url{ http://www.fdim.dk/sites/default/files/mediearkiv/rapporter/danskernes\_brug\_af\_internettet\_2012\_rapport.pdf}} om danskernes internetforbrug er adgang til internettet fra hjemmet steget fra 83\% til 92\% på blot 4 år - fra 2007 til 2011. I takt med den stigende grad af internetforbrug anvender annoncører de digitale medier til at ramme deres målgruppe.

Flere og flere virksomheder, interesseorganisationer og foreninger bliver derfor synlige på internettet med en hjemmeside eller profil på de sociale medier. De står alle overfor samme udfordring, nemlig hvordan man tjener penge på nettet. Den mest udbredte forretningsmodel er salg af annoncer på nettet. Der er to udfordringer ved at annoncere på nettet; den ene er den tekniske udfordring ved at selv skulle vedligeholde et bannersystem, og den anden er den økonomiske udfordring ved at skulle købe sig adgang til et bannersystem eller den nødvendige tekniske viden. Vores system er et forsøg på at løse disse to problemer.

\section{Easy Ad}

Projektet Easy Ad er et open source bannerreklamesystem. Formålet med Easy Ad er at udvikle et systemet der er nemt og gratis at bruge. Systemet skal kunne downloades af mindre ikke-kommercielle foreninger og interesseorganisationer, der hverken har de fornødne midler til at købe dyre systemer eller den tekniskeviden til at vedligeholde et sådan system.

Easy Ad skal udvikles til et firma ved navn Treu Media. Treu Media er specialiceret indenfor annoncesalg og en lang række andre medieudgivelser. Easy Ad skal gøre det nemt for Treu Media at administrer annoncer til deres mange kunder. Systemet skal derudover løse en lang række tekniske udfordringer som f.eks. hvordan man sikre med hvilken vægtning et banner bliver vist? Eller hvordan man undgår at samme banner bliver vist flere gang på samme hjemmeside.

Simon Shine er den tekniske kontaktperson og det er ham der sætter de tekniske krav til systemet. Jan Treu-Nielsen, Treu Media's ejer, vil være kontaktperson ved aftestning og opsætning af systemet. Projektet vil løbe fra marts og indtil ultimo juni. Systemet er en prototype og vil muligvis ikke være færdig udviklet ved projektets udgang. 

\section{IT-projektets formål og rammer}

Projektet har til formål 

\newpage

\section{FACTOR-kriteriet}

\large{\bf{F}}\normalsize{unctionality
\\
- Visning af bannere med forskellig vægtning}
\newline
\newline
\large{\bf{A}}\normalsize{pplication domain
\\
- afhængigt af senariet er det enten Jan eller en it-ansvarlig hos en organisation eller forening}
\newline
\newline
\large{\bf{C}}\normalsize{onditions\\
- Skal kunne bruges på tværs af browsere\\
- Skjult for brugere af hjemmesider\\
- Open source}
\newline
\newline
\large{\bf{T}}\normalsize{echnology\\
- Web-teknologi: html, css, php, javascript.\\
- Databaser}
\newline
\newline
\large{\bf{O}}\normalsize{bjects\\
- Bannere}
\newline
\newline
\large{\bf{R}}\normalsize{esponsibility\\
- Visning af bannere
- Holder styr på banner visning}

\newpage


\section{Kravspecifikation for IT-løsningen}
\subsection{Funktionelle og ikke-funktionelle krav}
\textbf{Funktionelle krav}
\\
EasyAd er et system, som gør det muligt at vedligeholde reklamer på en hjemmeside.

Systemet er opbygget sådan, at det kan være ejeren der administrerer systemet, eller det er en tredje part, der administrerer systemet. Dette gør, at man altid skal kunne logge på for at få adgang til systemet og for at kunne administrere reklamerne. Når man er logget på systemet, skal man oprette en hjemmeside (eller en kunde), hvor reklamen skal vises, og derefter skal man oprette en gruppe (eller en zone), som reklamen hører til. Til sidst skal man oprette selve reklamen og tilføje den til en gruppe.

Hjemmesiden (eller kunden), der oprettes kan være ens egen hjemmeside, eller det kan være kundens hjemmeside. I begge tilfælde skal hjemmesiden have adgang til systemet via en API, for at få vist en hjemmeside.

Grupperne bestemmer, hvor på siden, reklamen skal vises. Hvis man f.eks. kun ønsker at få en reklame vist på en underside, kan man lave en gruppe til undersiderne, mens man har en anden gruppe til forsiden.
\\
\\
\textbf{Ikke-funktionelle krav}
\\
Her er et nyt afsnit
\end{document}