\documentclass[a4paper,12pt]{article}

\usepackage[danish]{babel}
\usepackage[utf8]{inputenc}
\usepackage[pdftex]{graphicx}
\usepackage{subfigure}
\usepackage{usecases}
\usepackage{setspace}
\onehalfspacing
\usepackage{upquote}
\usepackage{color}
\definecolor{listinggray}{gray}{0.9}
\usepackage{listings}
\lstset{
	language=,
	literate=
		{æ}{{\ae}}1
		{ø}{{\o}}1
		{å}{{\aa}}1
		{Æ}{{\AE}}1
		{Ø}{{\O}}1
		{Å}{{\AA}}1,
	backgroundcolor=\color{listinggray},
	tabsize=3,
	rulecolor=,
	basicstyle=\scriptsize,
	aboveskip={1.5\baselineskip},
	columns=fixed,
	showstringspaces=false,
	extendedchars=true,
	breaklines=true,
	prebreak =\raisebox{0ex}[0ex][0ex]{\ensuremath{\hookleftarrow}},
	frame=single,
	showtabs=false,
	showspaces=false,
	showstringspaces=false,
	identifierstyle=\ttfamily,
	keywordstyle=\color[rgb]{0,0,1},
	commentstyle=\color[rgb]{0.133,0.545,0.133},
	stringstyle=\color[rgb]{0.627,0.126,0.941},
}
\usepackage[center,font=small,labelfont=bf,textfont=it]{caption}
\usepackage{enumerate}
\usepackage[numbers]{natbib}
\usepackage{hyperref}
\hypersetup{
	pdfborder = {0 0 0},
    colorlinks,
    citecolor=blue,
    filecolor=blue,
    linkcolor=blue,
    urlcolor=blue
}
\usepackage{parskip}
\setlength{\parindent}{15pt}

\begin{document}

\begin{titlepage}


\newcommand{\HRule}{\rule{\linewidth}{0.5mm}} % Defines a new command for the horizontal lines, change thickness here

\center % Center everything on the page

\textsc{\LARGE Københavns Universitet}\\[1.5cm] % Name of your university/college
\textsc{\Large Projektkursus Systemudvikling 2014}\\[0.5cm] % Major heading such as course name

\HRule \\[0.4cm]
{  \bfseries \large Open Source Bannerreklamesystem \\ \huge EasyAd}\\[0.4cm] % Title of your document
\HRule \\[1.5cm]

\begin{minipage}[t]{0.4\textwidth}
\begin{flushleft} \large
\emph{Projektgruppe:}


Signar Nielsen % Your name
\\
060585
\newline
\\
Aslak Niclasen
\\
060693
\newline
\\
Amanda Bergqvist
\\
171090
\end{flushleft}
\end{minipage}
~
\begin{minipage}[t]{0.4\textwidth}
\begin{flushright} \large
\emph{Instruktor:} \\
Aske Mottelson Clausen % Supervisor's Name
\end{flushright}
\end{minipage}\\[4cm]

{\large 14. maj 2014}\\[3cm] % Date, change the \today to a set date if you want to be precise

\end{titlepage}

\tableofcontents %generate table of content


\clearpage %clears float-buffer - inserting those missing - and starts new page

\section{Indledning}

Ifølge en rapport fra Danske Medier Research\footnote{\url{ http://www.fdim.dk/sites/default/files/mediearkiv/rapporter/danskernes\_brug\_af\_internettet\_2012\_rapport.pdf}} om danskernes internetforbrug er adgang til internettet fra hjemmet steget fra 83\% til 92\% på blot 4 år - fra 2007 til 2011. I takt med den stigende grad af internetforbrug anvender annoncører de digitale medier til at ramme deres målgruppe.

Flere og flere virksomheder, interesseorganisationer og foreninger bliver derfor synlige på internettet med en hjemmeside eller profil på de sociale medier. De står alle overfor samme udfordring, nemlig hvordan man tjener penge på nettet. Den mest udbredte forretningsmodel er salg af annoncer på nettet. Der er to udfordringer ved at annoncere på nettet; den ene er den tekniske udfordring ved at selv skulle vedligeholde et bannersystem, og den anden er den økonomiske udfordring ved at skulle købe sig adgang til et bannersystem eller den nødvendige tekniske viden. Vores system er et forsøg på at løse disse to problemer.

\section{Abstract}


\section{IT-projektets formål og rammer}


\subsection{FACTOR-kriteriet}

\large{\bf{F}}\normalsize{unctionality
\\
- Visning af bannere med forskellig vægtning
\newline
- Tidsbestemt/antalsbestemt bannervisning 
\newline
- Oprettelse af kunder 
\newline
- Oprettelse af grupper
\newline
- Oprettelse af bannere}
\newline
\newline
\large{\bf{A}}\normalsize{pplication domain
\\
- Medarbejdere tilknyttet Treu media}
\newline
\newline
\large{\bf{C}}\normalsize{onditions\\
- Skal kunne bruges på tværs af browsere\\
- Skjult for brugere af hjemmesider\\
- Open source}
\newline
\newline
\large{\bf{T}}\normalsize{echnology\\
- Web-teknologi: html, css, php, javascript.\\
- Databaser: MySQL\\
- Webserver: Apache}
\newline
\newline
\large{\bf{O}}\normalsize{bjects\\
- Bannere/Reklamer
\newline
- Administratorer
\newline
- Kunder
\newline
- Grupper}
\newline
\newline
\large{\bf{R}}\normalsize{esponsibility\\
- Holder styr på banner visning}

\section{Kravspecifikation for IT-løsningen}

\section{Systemdesign sammenfatning}

\section{Program- og systemtest}

\section{Brugergrænseflade og interaktionsdesign}

\section{Versionstyring}

\section{Projektsamarbejdet}

\section{Reviews}
\subsection{Review: No Silver Bullet}
\subsection{Review: Mocking-it-up or Hands-on The Future}
Artiklen som danskeren, Morten Kyng og svenskeren, Pelle Ehn, har skrevet omhandler det, som de kalder "participatory design". Artiklen er skrevet i 1991 og den beskriver, hvordan man kan designe ting med modeller (mockups), uden virkelige prototyper af det designede ting. De arbejde begge med et projekt, som kaldes Utopia. Deres opgave er at designe en arbejdsproces, som tager udgangspunkt i journalistens og typografens arbejde i 70erne, hvor linjerne for hvem laver hvad, er ved at blive lidt slørede, og hvor spændingen mellem journalisterne og typograferne er ret høj. De beskriver, hvordan man hurtigt og biligt kan opsætte en model af en arbejdeproces, som i dette tilfælde blandt andet består af computere, printere og software, hvor brugeren er involveret, ved hjælp af kartong og andre billige materialer. Fordelen ved brug af modeller er, at man nemt kan lave ændringer i designet, uden at det koster ekstra tid eller penge. En anden stor fordel ved disse modeller er, at alle har evnerne til at ændre i dem, fordi det kræver ikke viden inden programmering eller industriel design. Den største fordel ved modeller, fremfor beskrivelser på papir, er at den lægger op til hands-on erfaring og det opmuntrer involvering af brugeren.

Artiklen leder tankerne til Scrum, som er en metode inden softwareudvikling. Scrum er en del af de agile metoder inden for software udvikling. Scrum går ud på, at man har et Scrum team, som består af scrum master, produktejer og udviklings teamet. Scrum master har ansvaret for at produktet bliver leveret i henhold til kravene til kunden og til den tiden. Produkt ejeren repræsenterer typisk dem, der skal bruge produktet når det er færdigt. Produkt ejeren er med i hele forløbet, mens produktet bliver udviklet, på samme måde som modellerne og den "participatory design"-proces lagde op til. Produkt ejeren kan godt komme med ændringer til produktet mens det bliver udviklet, hvis der er behov for det, og Scrum udviklingsteamet udfører de ønskede ændringer. På denne måde sker der en iteration af krav til systemet og udvikling af kravene, hvorfor også det er en agil metode.

Vi kender også fra arbejde inden for softwareudvikling, at man bruger modeller, før man går i gang med at udvikle produktet. Signar har været med til at udvikle software, hvor man bruger papir, blyant og en saks til at lave modeller af produktet. Han har også været med til at bruge andre værktøjer, som f.eks. software, til at tegne modeller af produktet. Det er helt klart en fordel at bruge papir og blyant, når man skal lave den første skitse, fordi det er ekstremt hurtigt at ændre og det koster ikke noget. Det har også sin fordel at bruge software-værktøjer til at lave modeller, som f.eks. Photoshop eller Moqups.com, fordi det er et skridt tættere på det virklige produkt.

\end{document}