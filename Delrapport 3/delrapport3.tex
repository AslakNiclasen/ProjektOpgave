\documentclass[a4paper,12pt]{article}

\usepackage[danish]{babel}
\usepackage[utf8]{inputenc}
\usepackage[pdftex]{graphicx}
\usepackage{subfigure}
\usepackage{usecases}
\usepackage{setspace}
\onehalfspacing
\usepackage{upquote}
\usepackage{color}
\definecolor{listinggray}{gray}{0.9}
\usepackage{listings}
\lstset{
	language=,
	literate=
		{æ}{{\ae}}1
		{ø}{{\o}}1
		{å}{{\aa}}1
		{Æ}{{\AE}}1
		{Ø}{{\O}}1
		{Å}{{\AA}}1,
	backgroundcolor=\color{listinggray},
	tabsize=3,
	rulecolor=,
	basicstyle=\scriptsize,
	aboveskip={1.5\baselineskip},
	columns=fixed,
	showstringspaces=false,
	extendedchars=true,
	breaklines=true,
	prebreak =\raisebox{0ex}[0ex][0ex]{\ensuremath{\hookleftarrow}},
	frame=single,
	showtabs=false,
	showspaces=false,
	showstringspaces=false,
	identifierstyle=\ttfamily,
	keywordstyle=\color[rgb]{0,0,1},
	commentstyle=\color[rgb]{0.133,0.545,0.133},
	stringstyle=\color[rgb]{0.627,0.126,0.941},
}
\usepackage[center,font=small,labelfont=bf,textfont=it]{caption}
\usepackage{enumerate}
\usepackage[numbers]{natbib}
\usepackage{hyperref}
\hypersetup{
	pdfborder = {0 0 0},
    colorlinks,
    citecolor=blue,
    filecolor=blue,
    linkcolor=blue,
    urlcolor=blue
}
\usepackage{parskip}
\setlength{\parindent}{15pt}

\begin{document}

\begin{titlepage}


\newcommand{\HRule}{\rule{\linewidth}{0.5mm}} % Defines a new command for the horizontal lines, change thickness here

\center % Center everything on the page

\textsc{\LARGE Københavns Universitet}\\[1.5cm] % Name of your university/college
\textsc{\Large Projektkursus Systemudvikling 2014}\\[0.5cm] % Major heading such as course name

\HRule \\[0.4cm]
{  \bfseries \large Open Source Bannerreklamesystem \\ \huge EasyAd}\\[0.4cm] % Title of your document
\HRule \\[1.5cm]

\begin{minipage}[t]{0.4\textwidth}
\begin{flushleft} \large
\emph{Projektgruppe:}


Signar Nielsen % Your name
\\
060585
\newline
\\
Aslak Niclasen
\\
060693
\newline
\\
Amanda Bergqvist
\\
171090
\end{flushleft}
\end{minipage}
~
\begin{minipage}[t]{0.4\textwidth}
\begin{flushright} \large
\emph{Instruktor:} \\
Aske Mottelson Clausen % Supervisor's Name
\end{flushright}
\end{minipage}\\[4cm]

{\large 14. maj 2014}\\[3cm] % Date, change the \today to a set date if you want to be precise

\end{titlepage}

\tableofcontents %generate table of content


\clearpage %clears float-buffer - inserting those missing - and starts new page

\section{Abstract}

Projektet Easy Ad er et open source bannerreklamesystem. Formålet med Easy Ad er at udvikle et systemet der er nemt og gratis at bruge. Systemet skal kunne downloades af mindre ikke-kommercielle foreninger og interesseorganisationer, der hverken har de fornødne midler til at købe dyre systemer eller den tekniske viden til at vedligeholde et sådan system.

Easy Ad skal udvikles til et firma ved navn Treu Media. Treu Media er specialiceret indenfor annoncesalg og en lang række andre medieudgivelser. Easy Ad skal gøre det nemt for Treu Media at administrere annoncer til deres mange kunder. Systemet skal derudover løse en lang række tekniske udfordringer, som f.eks. hvordan man sikrer med hvilken vægtning et banner bliver vist. Eller hvordan man undgår at samme banner bliver vist flere gang på samme hjemmeside.

Simon Shine er den tekniske kontaktperson og det er ham der sætter de tekniske krav til systemet. Jan Treu-Nielsen, Treu Media's ejer, vil være kontaktperson ved aftesting og opsætning af systemet. Projektet vil løbe fra marts og indtil ultimo juni. Systemet er en prototype og vil muligvis ikke være færdig udviklet ved projektets udgang. 

\newpage

\section{Indledning}

Ifølge en rapport fra Danske Medier Research\footnote{\url{ http://www.fdim.dk/sites/default/files/mediearkiv/rapporter/danskernes\_brug\_af\_internettet\_2012\_rapport.pdf}} om danskernes internetforbrug er adgang til internettet fra hjemmet steget fra 83\% til 92\% på blot 4 år - fra 2007 til 2011. I takt med den stigende grad af internetforbrug anvender annoncører de digitale medier til at ramme deres målgruppe.

Flere og flere virksomheder, interesseorganisationer og foreninger bliver derfor synlige på internettet med en hjemmeside eller profil på de sociale medier. De står alle overfor samme udfordring, nemlig hvordan man tjener penge på nettet. Den mest udbredte forretningsmodel er salg af annoncer på nettet. Der er to udfordringer ved at annoncere på nettet; den ene er den tekniske udfordring ved at selv skulle vedligeholde et bannersystem, og den anden er den økonomiske udfordring ved at skulle købe sig adgang til et bannersystem eller den nødvendige tekniske viden. Vores system er et forsøg på at løse disse to problemer.

\section{IT-projektets formål og rammer}

Projektet har til formål at udvikle et open source bannerreklamesystem. Nedenfor ses projektets overordnede rammer og formål beskrevet i faktor-kriteriet. 

\subsection{FACTOR-kriteriet}

\large{\bf{F}}\normalsize{unctionality
\\
- Visning af bannere med forskellig vægtning
\newline
- Tidsbestemt/antalsbestemt bannervisning 
\newline
- Oprettelse af kunder 
\newline
- Oprettelse af grupper
\newline
- Oprettelse af bannere}
\newline
\newline
\large{\bf{A}}\normalsize{pplication domain
\\
- Medarbejdere tilknyttet Treu media}
\newline
\newline
\large{\bf{C}}\normalsize{onditions\\
- Skal kunne bruges på tværs af browsere\\
- Skjult for brugere af hjemmesider\\
- Open source}
\newline
\newline
\large{\bf{T}}\normalsize{echnology\\
- Web-teknologi: html, css, php, javascript.\\
- Databaser: MySQL\\
- Webserver: Apache}
\newline
\newline
\large{\bf{O}}\normalsize{bjects\\
- Bannere/Reklamer
\newline
- Administratorer
\newline
- Kunder
\newline
- Grupper}
\newline
\newline
\large{\bf{R}}\normalsize{esponsibility\\
- Holder styr på banner visning}

\section{Kravspecifikation for IT-løsningen}
\subsection{Funktionelle og ikke-funktionelle krav}
\textbf{Funktionelle krav}
\\
Funktionelle krav beskriver interaktionen mellem brugeren og systemet - uden at systemet fysisk, eller i praksis, er implementeret. Funktionelle krav siger noget om, hvordan systemet skal fungere, forventet input til systemet og forventet output fra systemet. De praktiske, eller fysiske, krav til systemet er beskrevet i ikke-funktionelle krav nedenunder.

EasyAd er et system, som gør det muligt at vedligeholde reklamer på ens hjemmeside. Systemet er opbygget sådan, at det kan være ejeren af en hjemmeside der administrerer systemet, eller det er en tredje part, der administrerer systemet. I vores tilfælde er det Treu Media, der administrerer reklamer på andres hjemmesider. Man altid skal kunne logge på for at få adgang til systemet og for at kunne administrere reklamerne. Når man er logget på systemet, skal man oprette en hjemmeside, hvor reklamen skal vises, og derefter skal man oprette en zone, som reklamen hører til. Dette kan f.eks. være specielle steder på hjemmesider. Til sidst skal man oprette selve reklamen og tilføje den til en zone.

Hjemmesiden, der oprettes får et automatisk genereret "access token", som er dens kodeord. Det er et krav, at hjemmesiden skal sende et access token med i sit request til systemet. Vi bruger access token - og andre mekanismer - til at identificere hjemmesiden, både for at undgå misbrug, men også for at kunne hente de rigtige reklamer.

Zonerne bestemmer, hvor på siden, reklamen skal vises. Hvis man f.eks. kun ønsker at få en reklame vist på en underside, kan man lave en zone til undersiderne, mens man har en anden zone til forsiden.
\\
\\
\textbf{Ikke-funktionelle krav}
\\
Ikke-funktionelle krav er direkte krav til systemet, hvor og hvordan det skal fungere. Ikke-funktionelle krav bliver også kaldt kvalitets krav, det vil sige krav til kvaliteten af det system, eller program, man udvikler. Man kan f.eks. stille krav til performance, pris, sikkerhed og mange andre ting. 

Da ikke-funktionelle krav er et vidt begreb, har vi valgt at bruge FURPS+ modellen, som bogen også bruger. FURPS står for Functionality, Usability, Reliability, Performance and Supportability. Plusset står for under-kategorier, som også bliver kaldt constraints.

Her er de ikke-funktionelle krav:
\begin{itemize}
	\item \textbf{Functionality:} Systemet skal kunne håndtere (oprette, ændre og vise) reklamer på en hjemmeside. Dette kræver selvfølgelig, at alle involverede parter har adgang til internettet. Det vil sige, systemet, hjemmesiden og brugeren af hjemmesiden.
	\item \textbf{Usability:} Formålet med systemet er, at det skal være nemt at bruge for alle involverede parter. Både administratorer, hjemmesider og brugere. Da det er meget svært at definere, hvad er "nemt" at bruge, laver vi nogle user-tests, for at sikre os, at det faktisk er nemt at bruge. Da tiden og resourcerne er begrænsede, vil vi ikke teste, om systemet er nemt at sætte op. Vi er nødt til at prioritere, og derfor vil vi i stedet teste, om systemet er nemt at bruge, efter det er blevet sat op. Dette vil vi teste sammen med Treu Media, hvor vi vil lave et tænk-højt eksperiment. Det er også vigtigt, at systemetet fungerer på alle moderne browsere (IE8+, Safari, Chrome og Firefox). Dette kan nemt testes i f.eks. browserstack.com - hvis der er tid til det, vil vi udføre sådan en test.
	\item \textbf{Reliability:} Denne del er lidt udenfor vores kontrol, da vi ikke kan styre, hvor og i hvilket miljø systemet vil blive sat op. Der skal ikke være nogen software-forhindring for, at systemet skal kunne være tilgængeligt 24 timer i døgnet 365 dage om året. Hvis vi antager, at serveren altid har forbindelse til internettet, så sætter dette først og fremmest krav til serveren, både hardware og software. Serveren skal være dimensioneret til at kunne håndtere trafikken og softwaren skal være designet på en hensigtsmæssig måde, så den ikke bruger unødvendige resourcer.
	\item \textbf{Performance:} Det er et krav til systemet, at svartiden er så lav som overhoved muligt. Der er flere faktorer, der spiller ind, men først og fremmest er det internetforbindelsen, serveren og selve softwaren. Brugen skal ikke bemærke, at han venter på, at reklamerne loader. Dette kan gøres på forskellige måder, f.eks. ved at loade reklamerne asynkront ved hjælp af XHR-objektet i JavaScript, så siden bliver mere brugervenlig.
	\item \textbf{Supportability:} Systemet kan understøttes af enhver, da det er open-source. Systemet bliver udviklet i PHP, hvilket også er et open-source sprog.
\end{itemize}

\textbf{Constraints:} Systemet er webbaseret og vil blive udviklet i PHP. Derudover benytter vi andre webteknologier, som f.eks. HTML, CSS, JavaScript (jQuery). Projektet er open-source, men hvis man vil udvikle, eller udvide, systemet, kan man gøre det i PHP. Vi har valgt at benytte en MySQL-database. I forhold til reklamerne skal man kunne uploade almindelige billeder, videosekvenser og flash-filer. Vi har ikke taget en endelig stilling til, hvilke filtyper man skal have lov til at uploade.

\section{Systemdesign sammenfatning}

\section{Program- og systemtest}

\section{Brugergrænseflade og interaktionsdesign}

\section{Versionstyring}

\section{Projektsamarbejdet}

\section{Reviews}
\subsection{Review: No Silver Bullet}
\subsection{Review: Mocking-it-up or Hands-on The Future}
Artiklen som danskeren, Morten Kyng og svenskeren, Pelle Ehn, har skrevet omhandler det, som de kalder "participatory design". Artiklen er skrevet i 1991 og den beskriver, hvordan man kan designe ting med modeller (mockups), uden virkelige prototyper af det designede ting. De arbejde begge med et projekt, som kaldes Utopia. Deres opgave er at designe en arbejdsproces, som tager udgangspunkt i journalistens og typografens arbejde i 70erne, hvor linjerne for hvem laver hvad, er ved at blive lidt slørede, og hvor spændingen mellem journalisterne og typograferne er ret høj. De beskriver, hvordan man hurtigt og biligt kan opsætte en model af en arbejdeproces, som i dette tilfælde blandt andet består af computere, printere og software, hvor brugeren er involveret, ved hjælp af kartong og andre billige materialer. Fordelen ved brug af modeller er, at man nemt kan lave ændringer i designet, uden at det koster ekstra tid eller penge. En anden stor fordel ved disse modeller er, at alle har evnerne til at ændre i dem, fordi det kræver ikke viden inden programmering eller industriel design. Den største fordel ved modeller, fremfor beskrivelser på papir, er at den lægger op til hands-on erfaring og det opmuntrer involvering af brugeren.

Artiklen leder tankerne til Scrum, som er en metode inden softwareudvikling. Scrum er en del af de agile metoder inden for software udvikling. Scrum går ud på, at man har et Scrum team, som består af scrum master, produktejer og udviklings teamet. Scrum master har ansvaret for at produktet bliver leveret i henhold til kravene til kunden og til den tiden. Produkt ejeren repræsenterer typisk dem, der skal bruge produktet når det er færdigt. Produkt ejeren er med i hele forløbet, mens produktet bliver udviklet, på samme måde som modellerne og den "participatory design"-proces lagde op til. Produkt ejeren kan godt komme med ændringer til produktet mens det bliver udviklet, hvis der er behov for det, og Scrum udviklingsteamet udfører de ønskede ændringer. På denne måde sker der en iteration af krav til systemet og udvikling af kravene, hvorfor også det er en agil metode.

Vi kender også fra arbejde inden for softwareudvikling, at man bruger modeller, før man går i gang med at udvikle produktet. Signar har været med til at udvikle software, hvor man bruger papir, blyant og en saks til at lave modeller af produktet. Han har også været med til at bruge andre værktøjer, som f.eks. software, til at tegne modeller af produktet. Det er helt klart en fordel at bruge papir og blyant, når man skal lave den første skitse, fordi det er ekstremt hurtigt at ændre og det koster ikke noget. Det har også sin fordel at bruge software-værktøjer til at lave modeller, som f.eks. Photoshop eller Moqups.com, fordi det er et skridt tættere på det virklige produkt.

\end{document}