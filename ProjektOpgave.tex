%article using font-size 12 and A4-size
\documentclass[a4paper,12pt]{article}

%use danish hyphenation and titles
%handle utf8-characters
\usepackage[danish]{babel}
\usepackage[utf8]{inputenc}
\usepackage[T1]{fontenc}

%for images
\usepackage[pdftex]{graphicx}

%allow nested figures
\usepackage{subfigure}

%control line spacing
\usepackage{setspace}
%\singlespacing
\onehalfspacing
%\doublespacing

%set margins
%\usepackage[margin=0.75in]{geometry}

%allows margin-notes
%good for work in progress papers
%\usepackage{marginnote}

%allows pretty quoting using ``'' or `'
\usepackage{upquote}

%two definitions of the color grey
\usepackage{color}
\definecolor{listinggray}{gray}{0.9}
%\definecolor{lbcolor}{rgb}{0.9,0.9,0.9}

%allows pretty source code
\usepackage{listings}
\lstset{
	language=,
	literate=
		{æ}{{\ae}}1
		{ø}{{\o}}1
		{å}{{\aa}}1
		{Æ}{{\AE}}1
		{Ø}{{\O}}1
		{Å}{{\AA}}1,
	backgroundcolor=\color{listinggray},
	tabsize=3,
	rulecolor=,
	basicstyle=\scriptsize,
	upquote=true,
	aboveskip={1.5\baselineskip},
	columns=fixed,
	showstringspaces=false,
	extendedchars=true,
	breaklines=true,
	prebreak =\raisebox{0ex}[0ex][0ex]{\ensuremath{\hookleftarrow}},
	frame=single,
	showtabs=false,
	showspaces=false,
	showstringspaces=false,
	identifierstyle=\ttfamily,
	keywordstyle=\color[rgb]{0,0,1},
	commentstyle=\color[rgb]{0.133,0.545,0.133},
	stringstyle=\color[rgb]{0.627,0.126,0.941},
}
%captions on listings
\usepackage[center,font=small,labelfont=bf,textfont=it]{caption}

%allows fancy enumeration
\usepackage{enumerate}

%allows use of the BibTex for the bibliography
\usepackage[numbers]{natbib}

%make references and URLs in the pdf to clickable links
\usepackage{hyperref}

%proper header formatting
\usepackage{fancyhdr}
\pagestyle{fancy}
\lhead[]{} %clear standard settings
\chead[]{} %clear standard settings
\rhead[]{\rightmark} %current section
\lfoot[]{} %clear standard settings
\cfoot[]{\thepage} %current page number 
\rfoot[]{} %clear standard settings

\begin{document}

\begin{titlepage}


\newcommand{\HRule}{\rule{\linewidth}{0.5mm}} % Defines a new command for the horizontal lines, change thickness here

\center % Center everything on the page

\textsc{\LARGE Københavns Universitet}\\[1.5cm] % Name of your university/college
\textsc{\Large Projektkursus Systemudvikling 2014}\\[0.5cm] % Major heading such as course name

\HRule \\[0.4cm]
{  \bfseries \large Open Source \\ \huge Bannerreklamesystem}\\[0.4cm] % Title of your document
\HRule \\[1.5cm]

\begin{minipage}[t]{0.4\textwidth}
\begin{flushleft} \large
\emph{Projektgruppe:}
\newline
Signar Nielsen
\\
060585
\newline
\\
Aslak Niclasen
\\
060693
\newline
\\
Amanda Bergqvist 171090
\end{flushleft}
\end{minipage}
~
\begin{minipage}[t]{0.4\textwidth}
\begin{flushright} \large
\emph{Instruktor:} \\
Aske Mottelson Clausen % Supervisor's Name
\end{flushright}
\end{minipage}\\[4cm]

\large{27. marts 2014}

\end{titlepage}

\tableofcontents %generate table of content


\clearpage %clears float-buffer - inserting those missing - and starts new page

\section{Problemformulering}

Ifølge en rapport fra Danske Medier Research\footnote{\url{ http://www.fdim.dk/sites/default/files/mediearkiv/rapporter/danskernes\_brug\_af\_internettet\_2012\_rapport.pdf}} om danskernes internetforbrug, er adgang til internettet fra hjemmet steget fra 83\% til 92\% på blot 4 år - fra 2007 til 2011. Udover adgang til internettet fra hjemmet, er danskerne i stigende grad online via mobile enheder. I takt med den stigende grad af internetforbrug, anvender annancører de digitale medier til at ramme deres målgruppe.

I takt med danskernes internet forbrug bliver flere og flere virksomheder, interesseorganisationer, foreninger synlige på internettet med en hjemmeside. De står alle sammen for samme udfordring, nemlig hvordan man tjener penge på nettet. Den mest udbredte forretningsmodel er salg af annoncer på nettet. Der er to udfordringer ved at annoncere på nettet, den ene er den tekniske udfordring ved at selv skulle vedligeholde et bannersystem, og den anden er den økonomiske udfordring ved at købe sig adgang til et bannersystem. Vores system løser disse to problemer.

Vores system er et opensource bannersystem, som gør det nemt at administrere annoncer på nettet, og fordi systemet er opensource, så er systemet gratis. Systemet er primært beregnet til to scenarier.

Scenarie et: Systemet bliver installeret hos tredje mand, som administrerer systemet og sælger annoncor til interessenter. Interessenten får en del af salget og annoncen bliver vist på interessentens digitale medie.

Scenarie to: Systemet bliver installeret direkte hos interessenten, som selv administrerer annoncerne på sit eget digitale medie.



https://github.com/AslakNiclasen/ProjektOpgave

\section{Indledende projektplan}

\section{Indledende skitse af system- og softwarearkitekturen}

Signar's tegning 

\section{Projektaftalen}

\section{Intern projektetablering}


Projektet er et open source bannerreklamesystem. Projektet skal kunne downloades og bruges af mindre ikke-kommercialle foreninger og interesseorganisationer, som ikke har de fornødne midler til at købe dyre systemer. Der findes på nuværende tidspunkt ikke noget pålideligt open source bannerreklamesystem, så projektet skal være en nyudvikling der er ment til at erstatte de nuværende brugte systemer. Klienten er en person ved navn Jan, der sælger reklame pladser på forskellige foreninger og organisationers hjemmesider. Projektet er i første omgang kun ment til at blive brugt af Jan, så Jan er den eneste interessent hos brugerorganisationen. Derudover er de forskellige foreninger og organisationer som Jan sælger reklame pladser for også interessenter, da de er intetresseret i at Jan kan sælge deres reklame pladser på bedst mulig vis. Projektet er ikke hundrede procent fastlagt, der er mange muligheder for udvidelse, så projektgruppen har stor mulighed for at påvirke projektet. Projektgruppen har endnu ikke haft et møde med Jan selv, men med en anden kontakt. Det er meningen at Jan kun vil være tilrådighed ved aftestning af projektet, eller hvis det opstår spørgsmål som kontaktpersonen ikke kan svare på. Projektgruppen er blevet informeret om at Jan selv har en server hvorpå systemet skal ligge, men at de derudover har stort set har ingen begrænsninger på hvordan projektet skal udvikles. 
\newline
Yderligere har projektet til formål at styrke projektgruppens erfaringer indenfor web-udwikling og databaser.
\newline
\newline
Projektgruppen er opdelt således at Signar har det overordnede overblik over projektetdelene. Han står derudover med ansvar for Javascript delen, da han har tidligere erfaring med dette. Aslak og Amanda fokusere på at styrke deres evner indenfor html, php og css, herunder bootstrap. Gruppen vil i fællesskab udvikle databasedelen.
\newline
Gruppen har valgt at kommunikere via github. 
Projektet kan finder under følgender URL: https://github.com/AslakNiclasen/ProjektOpgave 
\newline
\newline 

\section{FACTOR-kriteriet}

\large{\bf{F}}\normalsize{unctionality
\\
- Visning af bannere med forskellig vægtning}
\newline
\newline
\large{\bf{A}}\normalsize{pplication domain
\\
- Jan}
\newline
\newline
\large{\bf{C}}\normalsize{onditions\\
- Skal kunne bruges på tværs af browsere\\
- Skjult for brugere af hjemmesider\\
- Open source}
\newline
\newline
\large{\bf{T}}\normalsize{echnology\\
- Web-teknologi: html, css, php, javascript.\\
- Databaser}
\newline
\newline
\large{\bf{O}}\normalsize{bjects\\
- Bannere}
\newline
\newline
\large{\bf{R}}\normalsize{esponsibility\\
- Visning af bannere
- Holder styr på banner visning}

\newpage
\section{Bilag 1}
\begin{center}
\huge{\bf{Referat}}
\\
\normalsize{Projektgruppemøde}
\end{center}
Dato: 20/3-14
\newline
\newline
Tid: 11.30
\newline
\newline
Sted: Biocenteret
\newline
\newline
Inkaldte: Aslak, Signar og Amanda Bergqvist
\newline
\newline
Afbud: Ingen
\newline
\newline
\newline
\newline
Dagsorden:
\newline
\newline
1: Afklare spørgsmål vedrørende delrapport 1 og projektet.
\newline
\newline
2: Udele arbejdsopgaver i forhold til delrapport 1.
\newline
\newline
3: Aftale tidspunkt og dagsorden for næste møde.
\newline
\newline
\newline
\newline
Ad punkt 1:
Arbejdsfrekvens: Projektgruppen har aftalt at mødes hver torsdag, og mandag eller weekender om nødvendigt. 
\newline
\newline
Spørgsmål til instruktor: Literaturliste? fodnoter?
\newline
\newline
Ad punkt 2: Projektgruppen har aftalt at arbejde selvstændig på spørgsmålene indtil lørdag. På lørdag diskuteres spørgsmålene yderligere. Projektgruppen har aftalt at spørgsmålene skal være skrevet helt færdig til onsdag eftermiddag. 
Spørgsmålene er udeligeret som følger:   
\newline
\newline
(a) Signar 
\newline
\newline
(b) Aslak
\newline
\newline
(c) Signar 
\newline
\newline
(d) Aslak/ Amanda
\newline
\newline
(e) Amanda
\newline
\newline
FACTOR-kriteriet: Aslak
\newline
\newline
Referat: Amanda
\newline
\newline
Ad punkt 3: Projektgruppen har aftalt at næste møde er på DIKU på lørdag d. 22/3 kl 12. 
\newline
\newline
\newline
\newline
Referent: Amanda Bergqvist
\end{document}